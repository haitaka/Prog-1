\documentclass[aspectratio=169,14pt]{beamer}
\usepackage{common}
\usepackage{memoryviz}


\title{Программирование}
\subtitle{4 | Функции}
\author{a.glushko@g.nsu.ru}
\date{15 февраля 2021}


\begin{document}

    \begin{frame}
        \titlepage
    \end{frame}

    \begin{frame}{Функции}
        \begin{itemize}
            \item $f(x) = x^2$
            \item $h(x, y) = \sqrt{x^2 + y^2}$
            \item $k! = 1 \times 2 \times \dots \times k$
            \item<2> $f: X \rightarrow Y$
        \end{itemize}
    \end{frame}

    \begin{frame}[fragile]{Функции}
        \begin{itemize}
            \item $f(x) = x^2$
            \item $f: \mathbb{R} \rightarrow \mathbb{R}$
            \item<2->
                \begin{minted}{c}
                    double f(double x) {
                        return x * x;
                    }
                \end{minted}
        \end{itemize}
    \end{frame}

    \begin{frame}[fragile]{Заголовок функции}
        \begin{center}
            \begin{tikzpicture}
                \node at (0, 2) {$\op{fun}: \mathbb{Z} \times \mathbb{R} \rightarrow \mathbb{N}$};
                \node at (0, 0)  {\code[fontsize=\normalsize]{unsigned int fun(int x, double y) }};

                \begin{onlyenv}<2->
                    \draw [decorate,decoration={brace,amplitude=5pt,raise=10pt},xshift=0pt,yshift=-4pt,color=maincolor] (-4.5, 0) -- (-1.3, 0) node[midway,yshift=0.9em] (u) {};
                    \draw [decorate,decoration={brace,amplitude=3pt,raise=10pt},xshift=0pt,yshift=-4pt,color=maincolor] (0, 0) -- (0.8, 0) node[midway,yshift=0.9em] (i) {};
                    \draw [decorate,decoration={brace,amplitude=3pt,raise=10pt},xshift=0pt,yshift=-4pt,color=maincolor] (1.8, 0) -- (3.3, 0) node[midway,yshift=0.9em] (d) {};
                    \node [yshift=-0.3em] (z) at (-0.55, 2) {};
                    \draw [->,color=maincolor] (z) to[out=-90, in=90] (i);

                    \node [yshift=-0.3em] (r) at (0.45, 2) {};
                    \draw [->,color=maincolor] (r) to[out=-90, in=90] (d);

                    \node [yshift=-0.3em] (n) at (1.6, 2) {};
                    \draw [->,color=maincolor] (n) to[out=-90, in=90] (u);
                \end{onlyenv}
            \end{tikzpicture}
        \end{center}
    \end{frame}

    \begin{frame}[fragile]{void}
        \begin{minted}{c}
            void println(int x) {
                printf("%d\n", x);
            }
        \end{minted}
    \end{frame}

    \begin{frame}[fragile]{Факториал}
        \begin{onlyenv}<1>
            \begin{minted}{c}
                int factorial(int n) {
                    // ???
                }
            \end{minted}
        \end{onlyenv}
        \begin{onlyenv}<2>
            \begin{minted}{c}
                int factorial(int n) {
                    int res = 1;
                    for (int i = 0; i < n; ++i) {
                        res *= i;
                    }
                    return res;
                }
            \end{minted}
        \end{onlyenv}
        \begin{onlyenv}<3>
            \begin{minted}{c}
                int factorial(int n) {
                    if (n == 0) {
                        return 1;
                    }
                    return n * factorial(n - 1);
                }
            \end{minted}
        \end{onlyenv}
    \end{frame}

    \begin{frame}[fragile]{Локальные и глобальные переменные}
        \begin{columns}
            \begin{column}{0.5\textwidth}
                \begin{minted}{c}
                    #include <stdio.h>

                    int N = 3;

                    void printSquare(int x) {
                        int s = x * x;
                        printf("%d\n", s);

                        if (x < N) {
                            printSquare(x + 1);
                        }
                    }
                \end{minted}
            \end{column}
            \begin{column}{0.5\textwidth}
                \begin{onlyenv}<2>
                    \begin{itemize}
                        \item Где расположены \code{x} и \code{s}?
                    \end{itemize}
                \end{onlyenv}
                \begin{onlyenv}<3>
                    \begin{tikzpicture}[ scale=1 ]
%                        \draw (0, 0) rectangle (3, 1) node[pos=0.5] {qwe};
                    \end{tikzpicture}
                \end{onlyenv}
            \end{column}
        \end{columns}
    \end{frame}

    \qnaframe

\end{document}
