\documentclass[aspectratio=169,14pt]{beamer}
\usepackage{common}


\title{Программирование}
\subtitle{0 | Организационные вопросы}
\author{a.glushko@g.nsu.ru}
\date{12 февраля 2022}


\begin{document}

    \begin{frame}
        \titlepage
    \end{frame}

    \begin{frame}{Преподаватель}
        \begin{block}{Глушко Алексей Вячеславович}
            \begin{itemize}
                \item a.glushko@g.nsu.ru
                \item \includegraphics[height=\fontcharht\font`\B]{media/telegram128} \url{https://t.me/haitaka} % TODO icon
                \item +7~(996)~381~38~04
            \end{itemize}
        \end{block}
    \end{frame}

    \begin{frame}{Группа в Telegram}
        %\begin{center}
        \includegraphics[width=4cm]{media/telegram-invite-21125}

        \includegraphics[height=1em]{media/telegram128} \url{https://bit.ly/prog21125} % TODO icon
        %\end{center}
    \end{frame}

    \begin{frame}{Курс}
        \begin{itemize}
            \item Язык C
            \item Алгоритмы и структуры данных
                \begin{itemize}
                    \item На языке C
                \end{itemize}
        \end{itemize}
    \end{frame}

    \begin{frame}{Условия зачёта}
        \begin{columns}
            \column{0.43\textwidth}
                \begin{itemize}
                    \item максимум 100 баллов
                    \item $\geqslant 50$ --- оценка ``удовл''
                    \item $\geqslant 75$ --- оценка ``хорошо''
                    \item $\geqslant 85$ --- оценка ``отлично''
                \end{itemize}
            \column{0.6\textwidth}
                \begin{itemize}
                    \item Теория: 50 баллов
                    \item Практика: 50 баллов
                        \begin{itemize}
                            \item $5~\text{задач} \times 7~\text{баллов} = 35~\text{баллов}$
                            \item $\text{контрольная 1} = 10~\text{баллов}$
                            \item $\text{контрольная 2} = 5~\text{баллов}$
                        \end{itemize}
                \end{itemize}
        \end{columns}
    \end{frame}

    \begin{frame}{Задачи}
        \begin{itemize}
            \item 5 задач
            \item 3 уровня сложности
                \begin{itemize}
                    \item сложная --- 7 баллов
                    \item средняя --- 6 баллов
                    \item лёгкая --- 5 баллов
                \end{itemize}
            \item несколько вариантов каждого уровня
            \item Система тестирования \\
                \url{https://olympic.nsu.ru/nsuts-new}
        \end{itemize}
    \end{frame}

    \begin{frame}{Как сдать задачу}
        \begin{itemize}
            \item Выбрать вариант
            \item Написать решение
            \item Пройти автоматические тесты
            \item Пройти code review
            \item<2-> Пройти автоматические тесты
            \item<2-> Пройти code review
            \item<3-> Пройти автоматические тесты
            \item<3-> Пройти code review
            \item<4-> \dots
            \item<5-> Защитить решение на паре
        \end{itemize}
    \end{frame}

    \begin{frame}{Дедлайны}
        \begin{itemize}
            \item 2 недели на прохождение автоматических тестов и review
            \item Ещё 2 недели на защиту
            \item Штрафы
                \begin{itemize}
                    \item Задача отправлена (на последнее review) в первые 12 дней: нет штрафа
                    \item Задача отправлена в последние выходные до дедлайна: $-1$ балл
                    \item Задача отправлена в третью неделю: $-2$ балл
                    \item Задача отправлена в четвёртую неделю: $-3$ балла
                    \item После 4 недель задача не принимается
                \end{itemize}
        \end{itemize}
    \end{frame}

    \begin{frame}{Досрочный экзамен}
        \begin{itemize}
            \item Принимает семинарист
            \item Все задачи решены вовремя
            \item Во всех задачах взят сложный вариант
            \item Со всеми доп. заданиями
            \item Контрольные не пропущены
        \end{itemize}
    \end{frame}

\end{document}
