\documentclass[aspectratio=169,14pt]{beamer}
\usepackage{common}
\usepackage{memoryviz}


\title{Программирование}
\subtitle{Задачи}
\author{a.glushko@g.nsu.ru}
\date{\today}


\begin{document}

    \begin{frame}
        \titlepage
    \end{frame}

    \begin{frame}{Задания}
        \begin{enumerate}
            \item Сумматор из системы тестирования
            \item Написать программу, которая выводит на экран двоичное представление введенного неотрицательного целого числа в обратном порядке (от младшего разряда к старшему).
%            \item Считать N чисел, напечатать в обратном порядке (без массива).
%            \item Написать программу, которая проверяет простоту вводимых чисел (меньших N) с помощью решета Эратосфена.
%                Программа должна не заканчиваться сама, а ждать ввода очередного числа, проверять его простоту и выводить ответ.
%            \item
        \end{enumerate}
    \end{frame}

\end{document}
