\documentclass[aspectratio=169,14pt]{beamer}
\usepackage{common}


\title{Программирование}
\subtitle{1. Основные конструкции языка C}
\author{a.glushko@g.nsu.ru}
%\institute{University of ShareLaTeX}
\date{\today}


\begin{document}

    \begin{frame}
        \titlepage
    \end{frame}

    \begin{frame}{Язык C}
        \begin{itemize}
            \item K\&R C
            \item ANSI C
            \item C99
            \item \textbf{C11}
            \item C** % TODO
        \end{itemize}
    \end{frame}

    \begin{frame}{Язык C}
        \begin{itemize}
            \item Низкоуровневый
            \item Эффективный (?)
            \item Небезопасный
            \item Старый
        \end{itemize}
    \end{frame}

    \begin{frame}[fragile]{Hello, world!}
        \begin{columns}
            \column{0.5\textwidth}
                \begin{lstlisting}[extendedchars=\true]
                    #include<stdio.h>

                    int main() {
                        printf("Hello, world!");
                        return 0;
                    }
                \end{lstlisting}
            \column{0.5\textwidth}
                \begin{itemize}
                    \item Подключение библиотек
                    \item Точка входа
                \end{itemize}
        \end{columns}
    \end{frame}

    \begin{frame}[fragile]{Переменные}
    \end{frame}

    \begin{frame}[fragile]{Типы данных (примитивные)}
    \end{frame}

    \begin{frame}[fragile]{Арифметика}
    \end{frame}

    \begin{frame}[fragile]{Ввод/вывод}
    \end{frame}

    \begin{frame}[fragile]{Блоки и области видимости}
    \end{frame}

\end{document}
