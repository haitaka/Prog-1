\documentclass[aspectratio=169,14pt]{beamer}
\usepackage{common}
\usepackage{memoryviz}


\title{Программирование}
\subtitle{5 | Сложность}
\author{a.glushko@g.nsu.ru}
\date{15 марта 2021}


\begin{document}

    \begin{frame}
        \titlepage
    \end{frame}

    \begin{frame}{Сложность}
        \begin{itemize}
            \item Какой из двух алгоритмов лучше?
            \item Как долго выполняется алгоритм?
            \item Как много памяти он потребляет?
            \item<2> Что важнее?
        \end{itemize}
    \end{frame}

    \begin{frame}[fragile]{Поиск элемента в массиве}
        Для данного значения \code{x} и массива \code{arr} найти такой индекс \code{i}, \\что \code{arr[i] == x}.

        ~\\
        \begin{onlyenv}<1>
            \begin{minted}{c}
                int find(int arr[], int size, int x) {
                    ...
                }
            \end{minted}
        \end{onlyenv}
        \begin{onlyenv}<2>
            \begin{minted}{c}
                int find(int arr[], int size, int x) {
                    for (int i = 0; i < size; i++) {
                        if (arr[i] == x) {
                            return i;
                        }
                    }
                    return -1;
                }
            \end{minted}
        \end{onlyenv}
    \end{frame}

    \begin{frame}[fragile]{Поиск элемента в массиве}
        А если массив упорядочен?

        ~\\
        \begin{onlyenv}<2>
            \begin{minted}{c}
                int arr[] = {4, 8, 15, 16, 23, 42, 108};

                int find(int arr[], int size, int x) {
                    assert(isSorted(arr, size));
                    ...
                }
            \end{minted}
        \end{onlyenv}

        \begin{onlyenv}<3->
            \begin{enumerate}
                \item Если \code{size == 1}
                \item
                    \begin{onlyenv}<4->
                        Если \code{size == N > 1} \\
                        \only<5->{Положим, что задача решена для \code{size < N}}
                        ~\\
                        \only<6->{Выберем середину массива}
                        ~\\
                        \begin{tikzpicture}[ scale=0.8 ]
                            \draw [color=lightgray,draw=none] (1, 0)  rectangle (3, 1) node [pos=0.5] (p0) {\ldots};
                            \draw [color=lightgray] (3, 0)  rectangle (4, 1) node [pos=0.5] (p1) {15};
                            \draw [color=lightgray] (4, 0)  rectangle (5, 1) node [pos=0.5] (p2) {16};

                            \draw [color=lightgray] (6, 0)  rectangle (7, 1) node [pos=0.5] (a1) {42};
                            \draw [color=lightgray] (7, 0)  rectangle (8, 1) node [pos=0.5] (a2) {108};
                            \draw [color=lightgray,draw=none] (8, 0)  rectangle (10, 1) node [pos=0.5] (a0) {\ldots};

                            \draw [color=black,thick] (1, 0)  rectangle (5, 1);
                            \draw [color=black,thick] (5, 0)  rectangle (6, 1) node [pos=0.5] (m) {23};
                            \draw [color=black,thick] (6, 0)  rectangle (10, 1);

                            \begin{onlyenv}<7->
                                \draw [decorate,decoration={brace,amplitude=12pt,mirror,raise=2pt},xshift=0pt,
                                yshift=0pt] (1, 0) -- (5, 0) node [align=center,black,midway,yshift=-1cm] {здесь умеем искать};
                            \end{onlyenv}
                            \begin{onlyenv}<8->
                                \draw [decorate,decoration={brace,amplitude=12pt,mirror,raise=2pt},xshift=0pt,
                                yshift=0pt] (6, 0) -- (10, 0) node [align=center,black,midway,yshift=-1cm] {здесь тоже};
                            \end{onlyenv}
                        \end{tikzpicture}
                    \end{onlyenv}
            \end{enumerate}
        \end{onlyenv}
    \end{frame}

    \begin{frame}[fragile]{Поиск элемента в массиве}
        Для данного значения \code{x} и отсортированного массива \code{arr} найти такой индекс \code{i}, что
        \code{arr[i] == x}.

        ~\\
        \begin{onlyenv}<1>
            \begin{minted}{c}
                int find_binary(int arr[], int l, int r, int x) {
                    if (l > r) { return -1; }
                    int pivot = (l + r) / 2;
                    if (arr[pivot] == x) { return pivot; }
                    if (arr[pivot] > x) {
                        return find_binary(arr, l, pivot - 1, x);
                    }
                    return find_binary(arr, pivot + 1, r, x);
                }
            \end{minted}
        \end{onlyenv}
    \end{frame}

    \begin{frame}{Сравнение алгоритмов}
        \begin{itemize}
            \item<1-> Какой алгоритм лучше?
            \item<2-> Какой быстрее работает на отсортированном массиве?
            \item<3-> Как много операций приходится выполнять?
                \begin{itemize}
                    \item<4-> Зависит от размера входных данных (массива)
                    \item<5-> Зависит от характера входных данных
                \end{itemize}
        \end{itemize}
    \end{frame}

    \begin{frame}{Сравнение алгоритмов}
        \begin{block}{Как много операций приходится выполнять алгоритму?}
            \begin{itemize}
                \item<1-> Функция размера входных данных
                \item<1-> Функция распределения входных данных
                \begin{itemize}
                    \item<2-> В лучше случае
                    \item<3-> В худшем случае
                    \item<4-> В наиболее вероятном (среднем) случае
                \end{itemize}
            \end{itemize}
        \end{block}
    \end{frame}

    \begin{frame}{Сравнение алгоритмов}
        \begin{block}{Что аткое операция?}
            \begin{itemize}
                \item<1-> Арифметика
                \item<2-> Сравнения
                \item<3-> Чтение и запись памяти
                \item<4-> \dots
            \end{itemize}
        \end{block}
    \end{frame}

    \begin{frame}[fragile]{Поиск перебором}
        \begin{onlyenv}<1>
            \begin{itemize}
                \item Арифметика
                \item Сравнения
                \item Чтение и запись памяти
            \end{itemize}
        \end{onlyenv}
        \begin{minted}{c}
            int find(int arr[], int size, int x) {
                for (int i = 0; i < size; i++) {
                    if (arr[i] == x) {
                        return i;
                    }
                }
                return -1;
            }
        \end{minted}
        \begin{onlyenv}<2->
            \begin{itemize}
                \item<2-> 4 операции на итерацию
                \item<3-> От \code{1} до \code{size} итераций
                \item<6-> \code{(4 * size) + 2} операций в худшем случае
                \item<4-5> \code{4 * size} операций в худшем случае
                \item<5> Ещё инициализация \code{i} и \code{return}
            \end{itemize}
        \end{onlyenv}
    \end{frame}

    \begin{frame}[fragile]{Бинарный поиск}
        \begin{minted}{c}
            int find_binary(int arr[], int l, int r, int x) {
                if (l > r) { return -1; }
                int pivot = (l + r) / 2;
                if (arr[pivot] == x) { return pivot; }
                if (arr[pivot] > x) {
                    return find_binary(arr, l, pivot - 1, x);
                }
                return find_binary(arr, pivot + 1, r, x);
            }
        \end{minted}
        \begin{onlyenv}<2->
        \begin{itemize}
            \item<2-> Не менее 12 операции в каждом вызове!
            \item<3-> \code{[log_2(size)] + 1} вызовов
            \item<4-> \code{([log_2(size)] + 1) + 1} операций в худшем случае
        \end{itemize}
        \end{onlyenv}
    \end{frame}

    \begin{frame}{Сравнение сложности}
        \begin{itemize}
            \item<1-> \code{(4 * size) + 2}
            \item<1-> \code{([log_2(size)] + 1) + 1}
            \item<2-> Как сравнить?
            \item $O$
        \end{itemize}
    \end{frame}

    \begin{frame}{TODO}
        \begin{itemize}
            \item Подробнее про классы функций
            \item Средний случай
            \item Ёмкостная сложность
        \end{itemize}
    \end{frame}

    \qnaframe

\end{document}
