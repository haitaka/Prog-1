\documentclass[aspectratio=169,14pt]{beamer}
\usepackage{common}
\usepackage{memoryviz}


\title{Программирование}
\subtitle{3 | Статические массивы}
\author{a.glushko@g.nsu.ru}
\date{14 февраля 2022}


\begin{document}

    \begin{frame}
        \titlepage
    \end{frame}

    \begin{frame}[fragile,t]{Массив}
        Массив -- упорядоченный набор элементов одного типа.
        \begin{onlyenv}<2->
            \begin{center}
                \begin{tikzpicture}[ scale=0.8 ]
                    \memline{1}{15}

                    \memblock{2}{6}{x}
                    \only<-4>{\node[mem block] at (block x) {x}};
                    \only<5->{\node[mem block] at (block x) {\code{arr[0]}}};
                    \memaddr{2}{0x100}

                    \begin{onlyenv}<3->
                        \memblock{6}{10}{y}
                        \only<-4>{\node[mem block] at (block y) {y}};
                        \only<5->{\node[mem block] at (block y) {\code{arr[1]}}};
                        \memaddr{6}{0x104}
                    \end{onlyenv}

                    \begin{onlyenv}<4->
                        \memblock{10}{14}{z}
                        \only<-4>{\node[mem block] at (block z) {z}};
                        \only<5->{\node[mem block] at (block z) {\code{arr[2]}}};
                        \memaddr{10}{0x108}
                    \end{onlyenv}
                \end{tikzpicture}
            \end{center}
        \end{onlyenv}
        \only<5->{\hrule}
        \begin{onlyenv}<5>
            \begin{center}
                \begin{tikzpicture}
                    \node at (0, -1) {\code[fontsize=\Huge]{int arr[3];}};
                    \draw [decorate,decoration={brace,amplitude=12pt,mirror,raise=7pt},xshift=0pt,yshift=-4pt] (-3.5, -1.1) -- (-1.5, -1.1)
                        node [align=center,black,midway,yshift=-1.0cm] {тип элементов};
                    \draw [decorate,decoration={brace,amplitude=10pt,mirror,raise=10pt},xshift=0pt,yshift=-4pt] (1.5, -1.1) -- (2.2, -1.1)
                        node [align=center,black,midway,yshift=-1.38cm] {число элементов \\ (константа)};
                \end{tikzpicture}
            \end{center}
        \end{onlyenv}
        \begin{onlyenv}<6->
            \begin{columns}
                \lstset{basicstyle=\ttfamily\Large}
                \Large
                \begin{column}{0.5\textwidth}
                    \begin{center}
                        \begin{minipage}{0.8\textwidth}
                            {\color{maincolor}Запись:} \\
                            \code{arr[i] = 42;} \\
%                            \only<7->{\code{*(arr + i) = 42;}}
                        \end{minipage}
                    \end{center}
                \end{column}
                \begin{column}{0.5\textwidth}
                    \begin{center}
                        \begin{minipage}{0.9\textwidth}
                            {\color{maincolor}Чтение:} \\
                            \code{int x = arr[i];} \\
%                            \only<7->{\code{int x = *(arr + i);}}
                        \end{minipage}
                    \end{center}
                \end{column}
            \end{columns}
        \end{onlyenv}
    \end{frame}

    \begin{frame}[fragile]{Объявление массива}
        \begin{itemize}
            \item<1-> Размер статического массива --- константа
                \begin{itemize}
                    \item<2-> Целое число
                        \begin{minted}{c}
                            int arr[42];
                        \end{minted}
                    \item<3-> Константа препроцесора
                        \begin{minted}{c}
                            #define ARR_SIZE 42
                            int arr[ARR_SIZE];
                        \end{minted}
                \end{itemize}
        \end{itemize}
    \end{frame}

    \begin{frame}[fragile]{Инициализация массива}
        \begin{itemize}
            \item<1-> Без инициализации
                \begin{minted}{c}
                    int arr[4]; // {?, ?, ?, ?}
                \end{minted}
            \item<2-> С выводом размера
                \begin{minted}{c}
                    int arr[] = {4, 8}; // {4, 8}
                \end{minted}
            \item<3-> Частичная инициализация
                \begin{minted}{c}
                    int arr[4] = {15, 16}; // {15, 16, ?, ?}
                \end{minted}
            \item<4-> Инициализация строкой
                \begin{minted}{c}
                    char arr[] = "Hello"; // {'H', 'e', 'l', 'l', 'o', 0}
                \end{minted}
        \end{itemize}
    \end{frame}

    \begin{frame}[fragile,t]{Двумерные массивы}
        \begin{onlyenv}<1,2>
            \begin{minted}{c}
                #define N 3
                #define M 2

                int main() {
                    int arr[N][M] = {{4, 8}, {15, 16}, {23, 42}};
                    ...
                }
            \end{minted}
        \end{onlyenv}
        \begin{onlyenv}<3->
        \begin{minted}{c}
                int main() {
                    int arr[N][M] = {{4, 8}, {15, 16}, {23, 42}};

                    arr[0][0] == 4
                    arr[0][1] == 8
                    arr[2][1] == 42
                }
        \end{minted}
        \end{onlyenv}
        \begin{onlyenv}<2->
            \begin{center}
                \begin{tikzpicture}[ scale=1 ]
                    \memline{1}{9}

                    \draw [mem block,draw=none] (2,\membottom)  rectangle (3,\memtop)  node [pos=0.5] (a00) {4};
                    \draw [maincolor!50!black] (2,\membottom) grid (3,\memtop);
                    \draw [mem block,draw=none] (3,\membottom)  rectangle (4,\memtop)  node [pos=0.5] (a01) {8};
                    \draw [maincolor!50!black] (3,\membottom) grid (4,\memtop);
                    \draw [mem block,draw=none] (4,\membottom)  rectangle (5,\memtop)  node [pos=0.5] (a10) {15};
                    \draw [maincolor!50!black] (4,\membottom) grid (5,\memtop);
                    \draw [mem block,draw=none] (5,\membottom)  rectangle (6,\memtop)  node [pos=0.5] (a11) {16};
                    \draw [block byte lines] (5,\membottom) grid (6,\memtop);
                    \draw [mem block,draw=none] (6,\membottom)  rectangle (7,\memtop)  node [pos=0.5] (a20) {23};
                    \draw [block byte lines] (6,\membottom) grid (7,\memtop);
                    \draw [mem block,draw=none] (7,\membottom)  rectangle (8,\memtop)  node [pos=0.5] (a21) {42};
                    \draw [block byte lines] (7,\membottom) grid (8,\memtop);

                    \draw [mem block,fill=none,thick] (2,\membottom) rectangle (4,\memtop);
                    \draw [mem block,fill=none,thick] (4,\membottom) rectangle (6,\memtop);
                    \draw [mem block,fill=none,thick] (6,\membottom) rectangle (8,\memtop);

%                    \memaddr{2}{\footnotesize arr};
%                    \memaddr{4}{\footnotesize arr[1]};
%                    \memaddr{7}{\footnotesize arr[2][1]};
                \end{tikzpicture}
            \end{center}
        \end{onlyenv}
    \end{frame}

    \qnaframe

\end{document}
